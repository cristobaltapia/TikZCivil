%        File: doc-spa.tex
%     Created: dom abr 13 12:00  2014 C
% Last Change: dom abr 13 12:00  2014 C
%
\documentclass[11pt,letterpaper,oneside]{book}
\usepackage[left=3cm,right=2cm,top=1cm,bottom=1.5cm,includeheadfoot]{geometry}
\usepackage[spanish]{babel}
%\usepackage[latin1]{inputenc}
\usepackage[utf8x]{inputenc}
\usepackage[T1]{fontenc}
\usepackage{lmodern}
\usepackage{listings}
\usepackage{emp}
\usepackage{graphicx}
\usepackage{fancyhdr}
\usepackage{rccol}
\usepackage{verbatim}   % Comentarios multilinea
\usepackage{tabularx}
\usepackage{booktabs}
\usepackage{ifsym}
\usepackage{float}
\usepackage[font=small]{caption}
\usepackage{subcaption}
\setlength{\parindent}{1cm}
\usepackage{tikz}
\usepackage{subfig}
\usepackage{../StructuralAnalysis}

\begin{document}

\begin{titlepage}
  \title{Documentación para el uso de la librería tikzcivil}
  \author{Cristóbal Tapia\\
    \texttt{crtapia@gmail.com}
  }
  \date{\today}
  \maketitle
\end{titlepage}

\chapter{Dibujos relacionados con el análisis estructural}
\section{Dinámica}

%+++++++++++++++++++++++++++++++
% Se importa la libreria
%+++++++++++++++++++++++++++++++

\begin{figure}[!htp]
  \centering
  \begin{tikzpicture}[scale=1]
%Segundo argumento de MarcoDespl para empotramientos
%Tercer argumento de MarcoDespl para desplazamiento
    \MarcoDespl{0em,0em}{1}{2em}
% Segundo argumento de TMD es el desplazamiento
    \TMDAmor{4em,10em}{-2}
    \node at (3em,5em) {$ \displaystyle\frac{k_{j}}{2} $};
    \node at (13em,5em) {$ \displaystyle\frac{k_{j}}{2} $};
    \node at (9em,13.5em) {$m_{T}$};
    \node at (13.5em,9.3em) {$m_{j}$};
    \node at (5.0em,13.5em) {$k_{T}$};
    \node at (3.0em,10.6em) {$c_{T}$};
  \end{tikzpicture}
  \qquad
  \label{fig:rigi2}
\end{figure}



\chapter{Dibujos relacionados con la mecánica de suelos}
\end{document}


